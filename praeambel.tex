\usepackage{ifthen}

\usepackage[utf8]{inputenc} % Schriftsatz utf8 verwenden
\usepackage[T1]{fontenc} % T1-Fonts
\usepackage[latin,english,ngerman]{babel} % Deutsche Sonderzeichen und Silbentrennung 
\usepackage{mathpazo}
\usepackage[scaled=.95]{helvet}
\usepackage{courier}
\usepackage{marvosym} % diverse Symbole
\usepackage{wasysym} % diverse Symbole 
\usepackage{eulervm} % Mathematische Symbole 
\usepackage{amssymb} % Mathematische Symbole 
\usepackage{pifont} % einige Befehle brauchen pifont (Symbole)
\usepackage[babel,german=quotes]{csquotes}

\usepackage{tabularx} % bessere Tabellen
\usepackage[para]{threeparttable} %Fussnoten innerhalb von Tabellen
\usepackage{array} % wird von colortbl benötigt
\usepackage{colortbl} % erlaubt Farbige Zellen
\usepackage{soul}
\usepackage[table]{xcolor} % alternierende farbige Tabellenzeilen
\usepackage{booktabs} % Gestaltung der horizontalen Linien innerhalb einer Tabelle
\usepackage{multirow}
\usepackage{longtable}

\usepackage{xspace} 
\usepackage{geometry} 
\usepackage{setspace}
\usepackage{calc} 
\usepackage{microtype} 
\usepackage{setspace}
\usepackage{marginnote} % für Randnotizen

\usepackage[style=authortitle-dw,backend=biber]{biblatex}

\geometry{a4paper,left=25mm,right=25mm,top=15mm,bottom=15mm,includeheadfoot}
\clubpenalty = 10000
\widowpenalty = 10000
\displaywidowpenalty = 10000
\setlength{\parindent}{0pt}

\usepackage[headsepline,footsepline]{scrpage2}
\usepackage{totpages}

\usepackage{enumitem} %Anpassbare Enumerates/Itemizes

% Define the default values for these commands
\newcommand{\Universitaet}{}
\newcommand{\Title}{}
\newcommand{\Datum}{}
\newcommand{\Fakultaet}{}
\newcommand{\Semester}{}
\newcommand{\Dozent}{}
\newcommand{\Autor}{}
\newcommand{\Matrikelnr}{}
\newcommand{\Seminar}{}
\newcommand{\Kurzbeschreibung}{}
\newcommand{\Schluesselwoerter}{}
\newcommand{\Blank}{}

\makeatletter
% change the values if the user sets these commands
\newcommand{\SetTitle}[1]{\renewcommand{\Title}{#1}}
\newcommand{\SetUniversitaet}[1]{\renewcommand{\Universitaet}{#1}}
\newcommand{\SetFakultaet}[1]{\renewcommand{\Fakultaet}{#1}}
\newcommand{\SetSemester}[1]{\renewcommand{\Semester}{#1}}
\newcommand{\SetDozent}[1]{\renewcommand{\Dozent}{#1}}
\newcommand{\SetAutor}[1]{\renewcommand{\Autor}{#1}}
\newcommand{\SetMatrikelnr}[1]{\renewcommand{\Matrikelnr}{#1}}
\newcommand{\SetDatum}[1]{\renewcommand{\Datum}{#1}}
\newcommand{\SetSeminar}[1]{\renewcommand{\Seminar}{#1}}
\newcommand{\SetKurzbeschreibung}[1]{\renewcommand{\Kurzbeschreibung}{#1}}
\newcommand{\SetSchluesselwoerter}[1]{\renewcommand{\Schluesselwoerter}{#1}}
\newcommand{\SetLiteratur}[1]{\bibliography{#1}}
\makeatother
	
% Kopf und Fußbereich
\clearscrheadfoot
\pagestyle{scrheadings}
\setkomafont{pageheadfoot}{\footnotesize}

\makeatletter
	\ihead{\small \Autor \\ Matrikelnummer: \Matrikelnr}
	\chead{} 
	\ohead{\textbf{\small \Title} \\  \small \textbf{\Datum}}
	\ifoot{\Title} \cfoot{} \ofoot[\thepage{} / \pageref{TotPages}]{\thepage{} / \pageref{TotPages}}
\makeatother
\makeatletter
\defpagestyle{plainwfl}{
(0pt,0pt)
{} {} {
	\begin{minipage}[b]{90mm}
	\footnotesize
		\begin{flushleft}
			\textbf{\Universitaet} \\
			\textit{\Fakultaet} \\
			Seminar: \Seminar \\
			DozentIn: \Dozent
		\end{flushleft}
	\end{minipage}
	\hfill
	\begin{minipage}[b]{69mm} 
		\begin{flushright}
			AutorIn: \Autor \\
			Matrikelnummer: \Matrikelnr \\
			Datum: \today \\
			\Semester
		\end{flushright}
	\end{minipage}
}
(\textwidth,.4pt)
}{
(\textwidth,.4pt)
{} {} {
	\begin{minipage}[b]{79mm}
		\begin{flushleft}
			\Title
		\end{flushleft}
	\end{minipage}
	\hfill
	\begin{minipage}[b]{79mm} 
		\begin{flushright}
			\thepage{} / \pageref{TotPages}
		\end{flushright}
	\end{minipage}
}
(0pt,0pt)
}
\makeatother

% Hyperlinks und interne PDF-Verweise
\usepackage[
       % Farben fuer die Links
       colorlinks=true,         % Links erhalten Farben statt Kaeten
       urlcolor=black,    % \href{...}{...} external (URL)
       filecolor=black,  % \href{...} local file
       linkcolor=black,  % \ref{...} and \pageref{...}
       citecolor=black,
       menucolor=black,
       % Links
       raiselinks=true,			% calculate real height of the link
       backref=false,           % Backlinks im Literaturverzeichnis (section, slide, page, none)
       pagebackref=false,       % Backlinks im Literaturverzeichnis mit Seitenangabe
       verbose,
       hyperindex=true,         % backlinkex index
       linktocpage=true,        % Inhaltsverzeichnis verlinkt Seiten
       hyperfootnotes=true,     % Keine Links auf Fussnoten
       % Bookmarks
       bookmarks=false,         % Erzeugung von Bookmarks
       % Anchors
       plainpages=false,        % Anchors even on plain pages ?
       pageanchor=true,         % Pages are linkable
       pdfstartview=FitH,       % Dokument wird Fit Width geaefnet
       pdfpagemode=UseOutlines, % Bookmarks im Viewer anzeigen
       pdfpagelabels=true,      % set PDF page labels
]{hyperref}

% PDF Informationen
\makeatletter
\AtBeginDocument{
  \hypersetup{
    pdftitle={\Title},
    pdfauthor={\Autor},
    pdfsubject={\Kurzbeschreibung},
    pdfkeywords={\Schluesselwoerter}
  }
}
\makeatother

% Endnotes
\usepackage{endnotes}
\let\footnote=\endnote
\def\enotesize{\normalsize}
\renewcommand{\notesname}{Notes} 
\makeatletter
	\def\@makeentext#1{\theenmark) #1}
\makeatother

\newcommand{\notes}{
  \begingroup
  \parindent 0pt
  \leftmargin 2.5cm
  \parskip -1ex
  \def\enotesize{\normalsize}
  \theendnotes
  \endgroup
}
\usepackage{hyperendnotes} 

\makeatletter 
	% Redefine the sectioning commands \section \subsection and \subsubsection
	\renewcommand\section{\@startsection
		{section}{2}{0mm}{.6\baselineskip}{0.1\baselineskip}{\normalfont\large\textbf}
	}
	\renewcommand\subsection{\@startsection
		{subsection}{2}{0mm}{.6\baselineskip}{0.1\baselineskip}{\normalfont\normalsize\textbf}
	}
	\renewcommand\subsubsection{\@startsection
		{subsubsection}{2}{0mm}{.6\baselineskip}{0.1\baselineskip}{\normalfont\normalsize\textbf}
	}
\makeatother 

% für lange Zitate
\newenvironment{bquote}% 
   {\begin{quotation}\small}% 
   {\end{quotation}}% 
\SetBlockEnvironment{bquote}

% für kurze Zitate
\newenvironment{squote}% 
   {\begin{quote}\small}% 
   {\end{quote}}% 
\SetBlockEnvironment{squote}

% Hinweis und Warnung
\newcommand*{\Hinweis}[1]{\textbf{Hinweis:} #1}
\newcommand*{\Warnung}[1]{\textbf{Warnung:} #1}
\newcommand*{\Achtung}[1]{\textbf{Achtung:} #1}

\renewcommand\maketitle{
	\vspace*{0mm}\thispagestyle{plainwfl}
	\begin{center}
		\Large{\textbf{\Title}} \\
	\end{center}
	\onehalfspacing 
}   

\newcommand{\makeliteratur}{
	\singlespacing
	\nocite{*}
	\printbibliography
}